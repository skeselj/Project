%%%%%%%%%%%%%%%%%%%%%%%%%%%%%%%%%%%%%%%%%
% Short Sectioned Assignment
% LaTeX Template
% Version 1.0 (5/5/12)
%
% This template has been downloaded from:
% http://www.LaTeXTemplates.com
%
% Original author:
% Frits Wenneker (http://www.howtotex.com)
%
% License:
% CC BY-NC-SA 3.0 (http://creativecommons.org/licenses/by-nc-sa/3.0/)
%
%%%%%%%%%%%%%%%%%%%%%%%%%%%%%%%%%%%%%%%%%

%----------------------------------------------------------------------------------------
%	PACKAGES AND OTHER DOCUMENT CONFIGURATIONS
%----------------------------------------------------------------------------------------

\documentclass[paper=a4, fontsize=11pt]{scrartcl} % A4 paper and 11pt font size

\usepackage[T1]{fontenc} % Use 8-bit encoding that has 256 glyphs
\usepackage{fourier} % Use the Adobe Utopia font for the document - comment this line to return to the LaTeX default
\usepackage[english]{babel} % English language/hyphenation
\usepackage{amsmath,amsfonts,amsthm} % Math packages

\usepackage{lipsum} % Used for inserting dummy 'Lorem ipsum' text into the template

\usepackage{sectsty} % Allows customizing section commands
\allsectionsfont{\centering \normalfont\scshape} % Make all sections centered, the default font and small caps

\usepackage{fancyhdr} % Custom headers and footers
\pagestyle{fancyplain} % Makes all pages in the document conform to the custom headers and footers
\fancyhead{} % No page header - if you want one, create it in the same way as the footers below
\fancyfoot[L]{} % Empty left footer
\fancyfoot[C]{} % Empty center footer
\fancyfoot[R]{\thepage} % Page numbering for right footer
\renewcommand{\headrulewidth}{0pt} % Remove header underlines
\renewcommand{\footrulewidth}{0pt} % Remove footer underlines
\setlength{\headheight}{13.6pt} % Customize the height of the header

\numberwithin{equation}{section} % Number equations within sections (i.e. 1.1, 1.2, 2.1, 2.2 instead of 1, 2, 3, 4)
\numberwithin{figure}{section} % Number figures within sections (i.e. 1.1, 1.2, 2.1, 2.2 instead of 1, 2, 3, 4)
\numberwithin{table}{section} % Number tables within sections (i.e. 1.1, 1.2, 2.1, 2.2 instead of 1, 2, 3, 4)

\setlength\parindent{0pt} % Removes all indentation from paragraphs - comment this line for an assignment with lots of text

%----------------------------------------------------------------------------------------
%	TITLE SECTION
%----------------------------------------------------------------------------------------

\newcommand{\horrule}[1]{\rule{\linewidth}{#1}} % Create horizontal rule command with 1 argument of height

\title{	
\normalfont \normalsize 
\textsc{COS 333: Advanced Programming Techniques, Princeton University} \\ [25pt] % Your university, school and/or department name(s)
\horrule{0.5pt} \\[0.4cm] % Thin top horizontal rule
\huge SafeCity Design Document \\ % The assignment title
\horrule{2pt} \\[0.5cm] % Thick bottom horizontal rule
}

\author{
Eric He (ehe) \\
Kavin Sivakumar (ks16) \\
Stefan Keselj (skeselj) -- Liason}

\date{\normalsize\today} % Today's date or a custom date

\begin{document}

\maketitle % Print the title

%----------------------------------------------------------------------------------------
% OVERVIEW
%----------------------------------------------------------------------------------------

\section{Overview}


%----------------------------------------------------------------------------------------
% REQUIREMENTS AND TARGET AUDIENCES
%----------------------------------------------------------------------------------------

\section{Requirements and Target Audiences}

There is security in information. People can avoid danger and authorities can better combat it if key information is known. The police departments of most major American cities are continuously releasing vast amounts of crime data in hopes that people will use it to be better informed, but the average person does not gain insight from such raw and disorganized data dumps. SafeCity is a website which will bridge this knowledge gap by clearly and comprehensively displaying information about crimes in the US as they happen. \\

There have been multiple attempts at a tool like this, but none have managed to achieve the two attributes we think will differentiate our product: clear design and comprehensive information. CrimeMapping has sparse data, almost exclusively in Los Angeles (there are two crimes in the last month in New York). CrimeReports covers many cities, but the vast majority of crimes it reports are sexual assaults. SpotCrime has passable density over a fair number of cities, but the user can only see the crimes within a circle of a few miles in radius around a reference point, and zooming out does not display any more information. \\

SafeCity will be different from all these attempts in that it will display a large amount carefully curated crimes in an intuitive way according to type and time.This is a viable place for us to add value because the current resources are irresponsible about data completeness and careless with their user interfaces. Our vision will enable people with safety concerns about areas of their city, so all people at some point or another, to get much more meaningful insights than they would on any of these other websites. 

%----------------------------------------------------------------------------------------
% FUNCTIONALITY
%----------------------------------------------------------------------------------------

\section{Functionality}

The main focus of this application is to connect users with information about crimes happening in their cities. Each of the possible scenarios will be similar: a user will pan and select different areas on the map to the right and then interact with one of three different data representations on the control panel to the left. What differentiates these users is the type of data they will be interacting with, which comes in three forms: graphical, verbal, and quantitative.  

%------------------------------------------------

\subsection{Casual Passerby}

This user represents the people who stumble upon our site after having a transient interest in crime rates while surfing the web. The user will load the front page to find a large map taking up the rightmost two thirds of the screen, displaying all the crimes that happened in Manhattan in the last year as color coded bubbles. There will be a slider on the bottom of the map which will allow the user to play around with the time frame displayed on the map (maximum last ten years, minimum last week). The user can mouse over the map to find out details of each crime: time, type, and any people involved. To the left there will be a control panel summarizing the information in the map in different ways, which we hope will catch the user's eye for a few seconds. It will contain a line graph displaying the total number of crimes over time, a pie chart displaying all the different types of crimes, and a scrollable table containing all the crimes sorted chronologically. Once the user gets bored with Manhattan, he or she will type a different place, most likely his or her hometown, up in the search bar at the top of the screen and go through the same process and then leave.

%------------------------------------------------

\subsection{Active Member}

This user represents the people who have more lasting involvement with our application because they care about how the world views their community. This person will want the full benefits of being a member, so at some point they will click on the "Sign up" button on the top right and enter their username, password, full name, email address, and postal code (the last two will be verified). From then on they will either sign in through a login page or automatically sign in through their browser. This type of user has already explored the basic summaries mentioned above, and will spend more time on the second tab of the control panel: the impressions page. This page will be something like a Twitter feed, containing geotagged tweets involving safety or crime in an area, and containing a stack of 200 character testimonies people wrote about the safety of an area. Active members will be the primary contributors and curators of this information. They will write the testimonies about the areas in their city that they know, and they will upvote or downvote content as they feel fit to ensure that the most representative snippets are noticed. Periodically, these members will receive an email digest of what has been said about their area to invite them back.

%------------------------------------------------

\subsection{In-depth Researcher}

This user represents the people who are familiar to our site and come to it for more detailed and technical forms of information than those mentioned above. This user will occasionally check out the two tabs mentioned previously and be a member like the case above, but will spend most of their time viewing and downloading data from the third section of the control panel: the research tab. In this tab, the user will be able to download plain text files containing all the data used to make our application: all the times, types, and locations of crimes in an area and all the testimonials written about an area. The hope is that this user will use our information to gather further insights into how and why crime is happening. We will offer the option to upload documents in this tab, so that If the researcher uses our data to write a paper or report it can be made available to any other researchers looking to do something similar. Additionally, we will display our own take on using our data to research crime: a k-nearest-neighbors crime prediction algorithm which estimates the likelihood that certain crimes will happen in certain areas. This data will be overlaid to the right as a heatmap.

%----------------------------------------------------------------------------------------
% DESIGN
%----------------------------------------------------------------------------------------

\section{Design}

%------------------------------------------------

\subsection{Requirements and Target Audiences}

%------------------------------------------------

\subsubsection{Functionality}

%----------------------------------------------------------------------------------------
% TIMELINE
%----------------------------------------------------------------------------------------

\section{Timeline}

%------------------------------------------------

\subsection{Requirements and Target Audiences}

%------------------------------------------------

\subsubsection{Functionality}

%----------------------------------------------------------------------------------------
% RISKS AND OUTCOMES
%----------------------------------------------------------------------------------------

\section{Risks and Outcomes}

%------------------------------------------------

\subsection{Requirements and Target Audiences}

%------------------------------------------------

\subsubsection{Functionality}




\section{Rough Drafts}

Most users that come to our site will be people who had a spurious interest in the crime data of a particular city or area while surfing the web. These types of users have limited patience and interest in the subject, they need information displayed clearly without effort on their part. This type of user will open up our homepage to find a large map taking up two thirds of the screen zoomed in Manhattan by default, but there will be a search bar on top for navigation. The map will display crimes as basic bubbles color coded by time, and to the left there will be the three simplest ways to summarize the data for the user: a line graph displaying the crimes in the area over time, a pie chart displaying the types of crimes in the area, and a scrollable table containing the most recent crimes in that area. \\


Some users will have more lasting involvement with the site. For example, a community member concerned about buying or selling houses or a local politician who cares about his or her riding might want to get a sense of the character of an area from the perspective of people who have been there. For this purpose, there will be a second tab in the control panel featuring written "impressions" people have of the area (we label them as impressions and not reviews or reports to make them less formal). Each impression will be limited to 200 characters to ensure people get their point across effectively and to ensure compatibility with Twitter. Tweets about the area featuring key words like "crime" or "safe", and their relatives, will be displayed in the impressions tab as well. To contribute to the impressions, a user will haveBy scrolling through and contributing to the impressions tab, an active member of SafeCity will be able to define an area by more than just the cold numbers. 

\end{document}